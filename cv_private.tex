\documentclass[11pt,a4paper]{moderncv}

\moderncvtheme[blue]{classic} 
\usepackage[utf8]{inputenc}  %Windows 

%\usepackage[scale=0.975]{geometry}
\usepackage[top=0.5cm, bottom=0.5cm, left=0.5cm, right=0.5cm]{geometry}
\usepackage{graphicx}

\firstname{Anass}
\familyname{Mezroui}
\title{Data Scientist}   
\email{mezroui.anass7@gmail.com}                              
\extrainfo{}
\photo[50pt]{59342.jpg} 
\quote{Data Scientist passionné par la Data et IA}         
\makeatletter
\renewcommand*{\bibliographyitemlabel}{\@biblabel{\arabic{enumiv}}}
\makeatother

\usepackage{multibib}
\newcites{book,misc}{{Books},{Others}}

\nopagenumbers{}                         
\begin{document}
\maketitle
\section{Expérience professionnelle }
\cventry{08/2019 - Aujourd'hui}{Data Scientist}{Quinten Maroc}{(5 ans)}{}{Collaborer avec l'équipe DEV pour exploiter la valeur des données et créer des solutions data-driven, nettoyer et structurer les données brutes, développer des modèles prédictifs à l’aide d’algorithmes de machine learning pour anticiper des comportements futurs ou automatiser certaines tâches et faciliter l'intégration de ces modèles dans les pipelines de production.\newline{}
 \begin{itemize}
    \item Création de script \textbf{python} pour \textbf{Préparer} et \textbf{Transformer} des données textuelles ou tabulaires
    \item \textbf{Optimisation} des temps de calculs et de consommation des ressources des pipelines de \textbf{dataprep} existantes
    \item \textbf{Scraping} de différents fichiers \textbf{PDF} et de ressources \textbf{Web} pour collecter des \textbf{dataset} et créer des bases d'apprentissage pour des modèles de \textbf{deep learning}
    \item Créer des \textbf{Parsers} (analyseurs syntaxiques textuels) pour exécuter/évaluer des  \textbf{grammaires formelles LL(1)} prédéfinies par le besoin.
    \item Explorer et concevoir architectures de \textbf{Modélisation} des données pour créer des modèles \textbf{Prédictifs}
    \item Explorer différentes \textbf{Métriques} pour \textbf{Evaluer} la performance des modèles prédictifs
    \item Collaborer avec l'équipe DEV afin d'\textbf{Intégrer} et \textbf{Déployer} les modèles dans les projets existants
    \item Participer aux différentes réunions d'analyse et de conception \newline{}
\end{itemize}
}
\cventry{02/2024 - 07/2024}{Stagiaire Data Scientist}{Quinten Maroc}{}{(6 mois)}{Développement d'un système de scoring basé sur des techniques de  Text Mining pour créer un jeu de donnés à partir d'un dictionnaire de médicaments, élaborer une fonction pour annoter les données puis entrainer un modèle de machine learning SVM pour la régression sur la criticité des prescriptions médicales.}

\section{Éducation}
\cventry{2016 - 2019}{Formation d'ingénieur}{Ecole nationale d'informatique et d'analyse des systèmes (ENSIAS)}{Rabat}{}{Ingénierie Informatique et Systèmes Embarqués et mobiles}
\cventry{2014 - 2016}{Classe préparatoire aux grandes écoles (CPGE)}{Lycée Omar Ibn Abdelaziz}{Oujda}{}{MPSI/MP}
\cventry{2013 - 2014}{Baccalauréat}{Oued Eddahab}{Oujda}{}{Science mathématiques option A}

\section{Compétences techniques}
\cvline{}{Systèmes d'exploitation: \textbf{Linux (Debian, Ubuntu, Raspbian), Windows}}
\cvline{}{Langages de programmation: \textbf{Python , Java, C}}
\cvline{}{Bases de données: \textbf{SQL (PostgresSQL, MySQL, Oracle DB), NoSQL (Mongo DB)}}
\cvline{}{Data Transformation \& analysis: \textbf{Pandas, Numpy, PySpark, Dask}}
\cvline{}{Machine Learning: \textbf{PyTorch, Tensorflow, Scikit-learn, XGBoost, Transformers}}
\cvline{}{Model Serving: \textbf{Flask, TensorFlow Serving}}
\cvline{}{Model Monitoring: \textbf{TrochDrift, AlibiDetect}}
\cvline{}{MLOps: \textbf{MLflow}}
\cvline{}{Data visualisation: \textbf{Grafana, Seaborn, Matplotlib}}
\cvline{}{Indexing: \textbf{FAISS, Solr}}
\cvline{}{Image Processing: \textbf{OpenCV, Scikit-image, Pillow/PIL}}
\cvline{}{Autre compétences: \textbf{Docker, Git, Jupyter Notebooks, Selenium, PyParsing, PdfQuery, RegEx}\newline{}\newline{}\newline{}}

\section{Compétences générales}
\cvline{}{\textbf{Data Preparation, Data Visuallisation, Machine Learning, Statistical Analysis, Time-Series Forecasting, Classification, Regression, Clustering, Dimensionality Reduction, Natural Language Processing, Text Embeddings, Semantic Search, Text Parsing, Image Processing, Unit Testing, Web Scrapping, Agile Methodologies}}

\section{Algorithmes}
\cvline{}{Decision Trees, Random Forest, Support Vector Machines (\textbf{SVM}), 
Linear Regression, Logistic Regression, Gradient Boosting, K-Means,  Synthetic Minority Over-sampling Technique(\textbf{SMOTE}), Principal Component Analysis (\textbf{PCA}), Neural Networks (\textbf{CNN, RNN, LSTM, Autoencoders}), Bidirectional Encoder Representations from Transformers(\textbf{BERT})}

\section{Projets}
\cventry{}{Système de Recommandation}{}{}{}{Implémentation et déploiement d'un système de recommandation pour des produits cosmétiques sur un site e-commerce en se basant sur une approche de \textbf{Collaborative Filtering}. À partir d'une matrice d'interactions (\textbf{Utility Matrix}) entre les utilisateurs et les produits (les évaluations lors des achats), on a appliqué une \textbf{réduction de dimensionnalité} en utilisant des techniques comme la décomposition en valeurs singulières (\textbf{SVD}) pour améliorer l'efficacité et \textbf{réduire la complexité} des calculs, Ensuite, on a construit une \textbf{matrice de corrélation} (\textit{Pearson product-moment correlation coefficients}) entre les différents utilisateurs pour identifier les \textbf{utilisateurs similaires}. Cela a servi à recommander des \textbf{nouveaux produits} cosmétiques que d'autres \textbf{utilisateurs similaires} avaient déjà consultés ou achetés, optimisant ainsi l'expérience utilisateur. Le système a été évalué avec différentes métriques (\textbf{MSE}, $R^2$ \textbf{score,  Precision@k, Recall@k})}
\vspace{3mm}
\cventry{}{Prédiction des ventes trimestrielles}{}{}{}{Pour ce projet de prédiction des ventes trimestrielles, j'ai développé un modèle de \textbf{deep learning} en utilisant \textbf{TensorFlow}. Le modèle s'appuie sur une architecture hybrid \textbf{CNN-LSTM} combinant une couche \textbf{CNN} pour l'extraction des caractéristiques à partir des sous séquences (15 jours) suivie d'un \textbf{LSTM} pour capturer les \textbf{dépendances temporelles} des 30 derniers jours (2 sous séquences) de données historiques. J'ai mis en place un suivi détaillé des \textbf{performances} du modèle via \textbf{MLflow}, en loguant les \textbf{hyperparamètres}, les \textbf{métriques} et les \textbf{résultats} de chaque entraînement. Après le \textbf{déploiement} du modèle sur \textbf{MLflow}, j'ai intégré un mécanisme de détection de drift à l'aide de \textbf{TorchDrift}, et j'ai également automatisé la remontée des résultats de drift (\textbf{drift score, p\_value}) dans \textbf{MLflow} pour un \textbf{monitoring continu en production}. Le système a été évalué avec différentes métriques (\textbf{MSE}, $R^2$ score)}
\vspace{3mm}
\cventry{}{Moteur de recherche sémantique pour les textes de législation européenne}{}{}{}{Dans le cadre de ce projet, j'ai conçu et mis en œuvre un système capable de répondre à des \textbf{requêtes en langage naturel} en s'appuyant sur les textes du \textbf{EUR-Lex} (\textit{le point d'accès officiel et le plus complet aux documents législatifs de l'UE}). J'ai d'abord créé un dataset d'entraînement à l'aide de \textbf{BEIR} (\textit{Benchmarking Information Retrieval est un cadre de référence pour évaluer les modèles d'extraction d'informations qui propose différents outils pour générer des requêtes à partir d'un corpus}). Ensuite, j'ai effectué un \textbf{fine-tuning} d'un modèle pré-entraîné \textbf{Sentence-BERT} (\textbf{Bi-Encoder}) sur ces données afin d'améliorer la qualité des \textbf{vecteurs d'embeddings}. Enfin, j'ai  exploré plusieurs pistes de \textbf{déploiement} du système (\textbf{FAISS, Solr}) pour \text{indexer} la base de données des vecteurs et récupérer les \textbf{top 10 résultats} les plus \textbf{pertinents} en utilisant \textbf{cosine similarity} comme métrique.
}

\section{Langues}
\cvline{Arabe}{\textbf{Maternelle}}{}{}{}{}
\cvline{Anglais}{\textbf{Couramment} TOEFL iBT Novembre 2018 score: 86}{}{}{}{}
\cvline{Français}{\textbf{Couramment} TCF Octobre 2018 niveau C2}{}{}{}{}

\section{Centre d'intérêt}
\cvline{}{Natation, Raod trip à Moto.}

\end{document}


